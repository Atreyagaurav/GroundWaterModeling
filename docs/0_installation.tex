% Created 2022-04-25 Mon 15:19
% Intended LaTeX compiler: pdflatex
\documentclass[titlepage,12pt]{unisubmission}
	     \usepackage{booktabs}
	     \ClassCode{GEOL 6024}
\ClassName{GroundWater Modeling}
\ActivityType{Project}
\SubmissionType{Report}
\SubmissionNumber{0}
\SubmissionName{Installing MODFLOW for Flopy}
\Author{Gaurav Atreya}
\Mnumber{M14001485}
\Keywords{Groundwater,Modeling,Flopy}
\author{Gaurav Atreya}
\date{\today}
\title{Installing MODFLOW for Flopy}
\hypersetup{
 pdfauthor={Gaurav Atreya},
 pdftitle={Installing MODFLOW for Flopy},
 pdfkeywords={},
 pdfsubject={},
 pdfcreator={Emacs 28.1 (Org mode 9.5.2)}, 
 pdflang={English}}
\begin{document}

\maketitle
\setcounter{tocdepth}{2}
\tableofcontents
\section{How to install modflow for flopy}
\label{sec:org1bf6537}
Flopy is a python program which only handles the input and output from modflow, so it doesn't have the capacity to do the simulation itself. Hence you need to tell it where to find the modflow program so it can send the files to that program.

\subsection{Download as zip}
\label{sec:orgc1bcd97}
You can download the zip files for modflow From here:
\url{https://www.usgs.gov/software/modflow-6-usgs-modular-hydrologic-model}

The current version of MODFLOW 6 is version 6.2.2, released July 30, 2021.

The downloaded zip has following files when you unzip it.
\begin{minted}[]{bash}
unzip -l ~/Downloads/mf6.2.2.zip 
\end{minted}

\begin{minted}[]{text}
  mf6.2.2/
|-- bin		# contains mf6.exe executable
|-- doc		# contains pdfs
|-- examples	# contains a lot of examples
|-- make	# contains make files
|-- msvs	# contains Visual Studio solution/project
|-- src		# conaints the fortran source codes
|-- srcbmi	# same as prev
`-- utils	# contains utility programs/codes
\end{minted}

You only need the file \texttt{mf6.exe} in the \texttt{/bin} directory to connect to the flopy. 

\subsection{Compile it from source code}
\label{sec:org480e5fe}
You can compile the modflow code with fortran compiler to build the executables.

Using pymake from modflowpy makes it a bit easier. You can read the instructions here:

\url{https://github.com/modflowpy/pymake}


Run this in your terminal to install pymake.
\begin{minted}[]{bash}
pip install mfpymake  
\end{minted}

Then use this code in python to build all the modflow apps.
\begin{minted}[]{python}
import pymake

pymake.build_apps()
\end{minted}

You can choose to build only modflow 6.

Read the instructions in github for more information on that. Or read the documentation at \url{https://mfpymake.readthedocs.io/en/latest/}

\section{Flopy installation}
\label{sec:org26db465}
To install flopy you can use pip.
\begin{minted}[]{bash}
pip install flopy
\end{minted}

You can also install it through conda or similar package managers for python.

For models using zonebudget, there is a bug in flopy version 3.3.5. \href{https://github.com/modflowpy/flopy/issues/1395}{I brought it to the attention of the developers and they've fixed it}. But it's not yet available to the pip version 3.3.5. If the next version isn't yet released when you're running this code, then you can install the git's latest version with the following command.

\begin{minted}[]{bash}
pip install --upgrade git+https://github.com/modflowpy/flopy.git
\end{minted}


\section{Other Python libraries}
\label{sec:org690f59d}
To install all other libraries at once you can use the list in \texttt{requirements.txt} file.

Running this command in project root will install all requirements.
\begin{minted}[]{bash}
pip install -r requirements.txt
\end{minted}

The contents of the file has following libraries:

\begin{itemize}
\item numpy: For numerical analysis, used mostly for cell/layer properties
\item pandas: For reading/writing csv files
\item flopy: main flopy library
\item shapely: used for defining geometries and using GIS tools
\item matplotlib: used for plotting
\item scipy: for interpolation algorithm
\end{itemize}

The geopandas is optional, it's only used in data-gen for generating shapefiles (I think).
\begin{itemize}
\item geopandas: for reading writing shapefiles
\end{itemize}



\section{Exporting to vtk}
\label{sec:org3182fd5}
To export to vtk format, you need a vtk python library as well as working vtk in your computer, it was a bit complicated to setup so I'm not putting this here, if you have used vtk files and paraview program before you can probably figure it out.

For me (Arch Linux) it was simple enough to setup as the vtk was available in official repository, so I could do \texttt{sudo pacman -S vtk}. So depending on your OS you might have to find different ways. Please search it in the \href{https://vtk.org/}{official website of vtk}.
\end{document}